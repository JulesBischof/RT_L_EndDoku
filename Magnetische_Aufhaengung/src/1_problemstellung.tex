\chapter{Problemstellung}\label{chap:Problemstellung}
\section{Aufgabe 1}\label{sec:Aufgabe1}
	\subsection*{Blockschaltbild des geregelten Systems}\label{sub:block_regel}
	Das Blockschaltbild des geschlossenen Regelkreises ist in \autoref{dia:closed_loop} ersichtlich. Hierbei wird die Stecke wie auch das Stellglied  in $P$ zusammengefasst.
	$S$ bezeichnet dabei die Totzeit und den Fehler der durch den Laserdistanzmesser in das System eingeführt wird.


	\begin{figure}[h]
	\centering
	\begin{tikzpicture}[node distance=2cm]
		
      % Eingangsgröße
	\node (r) {};
	
	% Summenpunkt
	\node[circle, draw, right=of r] (sum) {};
	\node[below right=0.2pt and 0.1pt of sum] {$-$};
	\node[above left =2pt and 2pt of sum] {$+$};
	
	% Regler
	\node[draw, rectangle, minimum width=1cm, minimum height=1cm,,right=of sum] (controller) {C $\frac{h}{i}$};
	
	% Strecke
	\node[draw, rectangle,  minimum width=1cm, minimum height=1cm, right=of controller] (plant) {P};

	% Sensor
	\node[draw, rectangle,  minimum width=1cm, minimum height=1cm, below=0.5 of controller] (sensor) {S};
	
	% Ausgang
	\node[right=of plant] (y) {};
	
	% Verbindungslinien
	\draw[->] (r) -- (sum) node[midway, above]  {$x_{ref} (t) $};
	\draw[->] (sum) -- (controller) node[midway, above] {$x_{\text{err}}$};
	\draw[->] (controller) -- (plant) node[midway, above] {$i(t)$};
	\draw[->] (plant) -- (y)  node[midway, above]  {$x(t)$};
	
	% Rückführung: von Linie abzweigen und unten herum zurück
	\coordinate (tap) at ($(plant)!0.5!(y)$); % Punkt auf der Linie zwischen plant und y
	\coordinate (feedback) at ($(sum)+(0,-2)$); % Punkt unterhalb des Summenpunkts
	
	\draw[->] (tap) |- (sensor) -| (sum);
		
	\end{tikzpicture}
	\caption{Geschlossener Regelkreis}
	\label{dia:closed_loop}
\end{figure}

	
\section{Aufgabe 2}\label{sec:Aufgabe2}
	\subsection*{Blockschaltbild des geregelten Systems mit Vorsteuerung}\label{sub:block_regel_FF}
	Das Blockschaltbild aus \autoref{sec:Aufgabe1} wird in \autoref{dia:closed_loop_FF} um eine Vorsteuerung $FF$ erweitert.


	\begin{figure}[h]
	\centering
	\begin{tikzpicture}[node distance=2cm]
		
      % Eingangsgröße
	\node (r) {};
	
	% Summenpunkt
	\node[circle, draw, right=of r] (sum) {};
	\node[below right=0.2pt and 0.2pt of sum] {$-$};
	\node[below left =-5pt and 0.2pt of sum] {$+$};

	% Summenpunkt2
	\node[circle, draw, right=of controller] (sum2) {};
	\node[above right =0.2pt and -5pt of sum2] {$+$};
	\node[below left =-5pt and 0.2pt of sum2] {$+$};
	
	% Regler
	\node[draw, rectangle, minimum width=1cm, minimum height=1cm, right=of sum] (controller) {C $\frac{h}{i}$};
	
	% Strecke
	\node[draw, rectangle,  minimum width=1cm, minimum height=1cm, right=of sum2] (plant) {P};

	% Sensor
	\node[draw, rectangle,  minimum width=1cm, minimum height=1cm, below=0.5 of controller] (sensor) {S};

	% Vorsteuerung
	\node[draw, rectangle, minimum width=1cm, minimum height=1cm, above=0.5 of controller] (prescaler) {FF$\frac{h}{i}$};


	% Ausgang
	\node[right=of plant] (y) {};
	
	% Verbindungslinien
	\draw[->] (r) -- (sum) node[midway, above]  {$x_{ref} (t) $};
	\draw[->] (sum) -- (controller) node[midway, above] {$x_{\text{err}}$};
	\draw[->] (controller) -- (sum2) node[midway, above] {$i_C(t)$};
	\draw[->] (sum2) -- (plant) node[midway, above] {$i(t)$};
	\draw[->] (plant) -- (y)  node[midway, above]  {$x(t)$};
	
	% Rückführung: von Linie abzweigen und unten herum zurück
	\coordinate (tap) at ($(plant)!0.5!(y)$); % Punkt auf der Linie zwischen plant und y
	\coordinate (feedback) at ($(sum)+(0,-2)$); % Punkt unterhalb des Summenpunkts
	\coordinate (feedforward) at ($(sum2)+(0,1)$); % Punkt unterhalb des Summenpunkts
	\coordinate (tap_pre) at ($(sum)+(-0.5,0)$); % Punkt vor dem Summenpunkt
	
	\draw[->] (tap) |- (sensor) -| (sum);

	\draw[->] (tap_pre) |-  (prescaler) -| (feedforward) node[midway, above]{$i_{FF}(t)$} -- (sum2) ;
		
	\end{tikzpicture}
	\caption{Geschlossener Regelkreis erweitert mit einer Vorsteuerung}
	\label{dia:closed_loop_FF}
\end{figure}

