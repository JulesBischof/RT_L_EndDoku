\newcommand{\laplace}{%
  \tikz[baseline={(base)}, inner sep=0pt, outer sep=0pt]{
    % Solid and hollow dots as nodes
    \node[circle, fill, inner sep=1.5pt] (solid) at (0,0) {};
    \node[draw, circle, inner sep=1.5pt] (hollow) at (0,.5) {};
    % Connecting line
    \draw (solid) -- (hollow);
    % Baseline anchor
    \coordinate (base) at (0,0);
  }
}

\chapter{Modellierung}\label{chap:Modellierung}
\section{Aufgabe 3}\label{sec:Aufgabe3}
	\subsection*{Bewegungsdifferentialgleichung}\label{sub:diffeq}
	Aus der gegebenen Bewegungsdifferentialgleichung und der, mittels eines Polynoms dritten Grades approximierten, statischen Kennlinie $i_o(x) = a_i + b_ix +c_ix^2 + d_ix^3$ ergibt sich für die Bewegungsdifferentialgleichung \ref{eq:bewegungsgleichung}. 
	
	\begin{align}\label{eq:bewegungsgleichung}
		\ddot{x} &= g - g \cdot \frac{i^2}{i_0^2(x)} \nonumber\\
		\ddot{x} &= g - g \cdot \frac{i^2}{(a_i + b_ix +c_ix^2 + d_ix^3)^2}
	\end{align} 

\section{Aufgabe 4}\label{sec:Aufgabe4}
	\subsection*{Linearisierung}\label{sub:diffeq_lin}
	Zur Linearisierung der Bewegungsdifferentialgleichung wird folgende Struktur der linearisierten Differentialgleichung eingesetzt:

	\begin{align}\label{eq:lin_eq} 
		\Delta\ddot{x} &= k_x\Delta x + k_i\Delta i + k_s \Delta F_s
	\end{align} 

	Die Faktoren $k_x$, $k_i$ und $k_s$ werden aus \autoref{eq:bewegungsgleichung} in einem Arbeitspunkt $x_o$ und $\bar{i}$ bestimmt. Dazu wird \autoref{eq:bewegungsgleichung} jeweils nach $\delta x$, $\delta i$ und $\delta F_s$ abgeleitet. Da für den Versuch die Störkraft $F_s = 0$ angenommen wird, muss $k_s$ nocht ermittelt werden. 


	\begin{align} \label{eq:kx}
		k_x &= \left. \frac{\delta}{\delta x}  \left( g-g\frac{i^2}{(a_i+b_ix+c_ix^2+d_ix^3)^2} \right) \right|_{x_0, \bar{i}} \nonumber \\
			&= \left. \frac{\delta}{\delta x} \frac{-gi^2}{(a_i+b_ix+c_ix^2+d_ix^3)^2} \right|_{x_0,\bar{i}}  \nonumber \\
			&= \frac{2g\bar{i}^2(3d_ix_0^2 + 2c_ix_0+b)}{(d_ix_0^3+c_ix_0^2+b_ix_0+a)^3} 
	\end{align} 

	\begin{align} \label{eq:ki}
		k_i &= \left. \frac{\delta}{\delta i}  \left( g-g\frac{i^2}{(a_i+b_ix+c_ix^2+d_ix^3)^2} \right) \right|_{x_0, \bar{i}} \nonumber \\
			&= \left. \frac{\delta}{\delta i} \frac{-gi^2}{(a_i+b_ix+c_ix^2+d_ix^3)^2} \right|_{x_0,\bar{i}}  \nonumber \\
			&= \frac{-2g\bar{i}}{(a_i + b_ix_0 +c_ix_0^2 + d_ix_0^3)^2}
	\end{align} 

	Werden nun \autoref{eq:kx} und \ref{eq:ki} in \autoref{eq:lin_eq} eingesetzt ergibt sich:

	\begin{align}\label{eq:lin_eq_k} 
		\Delta\ddot{x} &= \frac{2g\bar{i}^2(3d_ix_0^2 + 2c_ix_0+b)}{(d_ix_0^3+c_ix_0^2+b_ix_0+a)^3}  \cdot \Delta x + \frac{-2g\bar{i}}{(a_i + b_ix_0 +c_ix_0^2 + d_ix_0^3)^2} \cdot \Delta i 
	\end{align} 

\section{Aufgabe 5}\label{sec:Aufgabe5}
	\subsection*{Übertragungsfunktion $G_{\mathrm{Strecke}}$ }\label{sub:transfer_Gstrecke}
	Um die Übertragungsfunktion der Prozesstrecke $G_{\mathrm{Strecke}}$ zu finden, kann erneut \autoref{eq:lin_eq} bzw. \autoref{eq:lin_eq_k} genutzt werden.


	\begin{align}\label{eq:G_strecke} 
		\Delta \ddot{x} &= k_x \Delta x + k_i \Delta i \nonumber\\
		\laplace \nonumber \\
		s^2 \Delta X &= k_x \Delta X + k_i \Delta I  \notag  \\
		G_{\mathrm{Strecke}}(s) &= \frac{\Delta X}{\Delta I} = \frac{k_i}{s^2-kx} \nonumber\\
		G_{\mathrm{Strecke}}(s) &= \frac{\frac{-2g\bar{i}}{(a_i + b_ix_0 +c_ix_0^2 + d_ix_0^3)^2}}{s^2-\frac{2g\bar{i}^2(3d_ix_0^2 + 2c_ix_0+b)}{(d_ix_0^3+c_ix_0^2+b_ix_0+a)^3}}
	\end{align}

\newpage 

\section{Aufgabe 6}\label{sec:Aufgabe6}
	\subsection*{Statische Vorsteuerung}\label{sub:statisch_ff}
	Der Einsatz einer statischen Vorsteuerung besteht darin, dass kein Regelfehler $e(t)$ nötig ist und somit nicht auf die Rückmeldung des Sensors gewartet werden muss. Die Regelung wird schneller.


\section{Aufgabe 7}\label{sec:Aufgabe6}
	\subsection*{Statische Vorsteuerung}\label{sub:statisch_ff}
	Die statischen Vorsteuerungen $V_{\mathrm{L}}$ und 	$V_{\mathrm{NL}}$ ergeben sich aus $G_{\mathrm{Strecke}}^{-1}(s)$ wie folgt:
	
	\begin{align}\label{eq:VL} 
		V_{\mathrm{L}} &= \left. G_{\mathrm{Strecke}}^{-1}(s) \right |_{s = 0} = \left. \frac{s^2-kx}{k_i} \right |_{s = 0}  \nonumber\\
		V_{\mathrm{L}} &= \frac{-kx}{k_i}
	\end{align} 
	\begin{align}\label{eq:VL} 
		V_{\mathrm{NL}} &= \left. G_{\mathrm{Strecke}}^{-1}(s) \right |_{s = 0} = \left. \frac{s^2-kx}{k_i} \right |_{s = 0}  \nonumber\\
		V_{\mathrm{NL}} &= |d + cx_0^1 + bx_0^2 + ax_0^3|
	\end{align} 

\section{Aufgabe 8}\label{sec:Aufgabe8}
	\subsection*{Übertragungsfunktion $G_{\mathrm{Stell}}$ }\label{sub:transfer_Gstell}
	Mit der gegebenen Differentialgleichung des Stellglieds $u = Ri +L\frac{\mathrm{d}i}{\mathrm{d}t}$ kann die Übertragungsfunktion des Stellglieds $G_{\mathrm{Stell}}$ wie folgt gefunden werden:

	\begin{align}\label{eq:G_stell} 
		u &= Ri +L\frac{\mathrm{d}i}{\mathrm{d}t} \nonumber \\
		\laplace \nonumber \\
		U &= RI + sLI \nonumber \\
		G_{\mathrm{Stell}}(s) &= \frac{I}{U} = \frac{1}{R+Ls} = \frac{1}{1+\frac{L}{R}s}
	\end{align}
\newpage
\section{Aufgaben 9 und 10}\label{sec:Aufgabe9_10}
\subsection*{Nachstellzeit $T_{\mathrm{i}}$ des Stellgliedreglers und Übertragungsfunktion des Stellglieds}\label{sub:transfer_Ti_Cstell}

Der Regler des Stellglieds besteht aus einem PI-Regler mit Übertragungsfunktion $C_{\mathrm{Stell}} = K_{PS} \left(1+\frac{1}{s+T_i} \right)$. Damit ergibt sich für das geregelte Stellglied: 
	\begin{align}\label{eq:G_stell_closed} 
		G_{\mathrm{StellClosed}} &= \frac{L_{\mathrm{Stell}}}{1 + L_{\mathrm{Stell}}} = \frac{C_{ \mathrm{Stell}}G_{\mathrm{Stell}} }{1 +  C_{\mathrm{Stell}}G_{\mathrm{Stell}}}	
	\end{align}

	Hierbei zeigt sich, dass mit $T_{s} = \frac{L}{R} = T_{i}$ ein PT-1 erreicht werden kann:
	\begin{align}\label{eq:L_stell} 
		L_{\mathrm{Stell}}  &= K_{PS} \left(1+\frac{1}{s+T_i} \right) \cdot \frac{K_{S}}{1 + T_{S}s} \nonumber \\
							&= K_{PS} K_{S} \cdot \frac{1 + T_{S}}{T_{S}\left( 1 + T_{S} \right)} = \frac{K_{PS} K_{S} }{T_{S}}
	\end{align}

	Wird nun  \autoref{eq:L_stell} in \autoref{eq:G_stell_closed} eingesetzt ergibt sich die Übertragungsfunktion:
	\begin{align}\label{eq:G_stell_closed2} 
		G_{\mathrm{StellClosed}} &= \frac{K_{PS} K_{S}}{T_{s}s + K_{PS} K_{\mathrm{S}}}
	\end{align}



\section{Aufgabe 11}\label{sec:Aufgabe11}
	\subsection*{Übertragungsfunktion $G_{\mathrm{Sensor}}$ }\label{sub:transfer_Gsensor}
	Wir wissen, dass der sensor eine Totzeit $T_{t\mathrm{Sensor}}$ von $2.5 \si{\milli\second}$ aufweist. Somit kann seine Übertragungsfunktion $G_{\mathrm{Sensor}}$ mittels eines PT-1 Glieds beschrieben werden:

	\begin{align}\label{eq:G_sens} 
	G_{\mathrm{Sensor}} &= \frac{\Delta X_{m}}{\Delta X} = \frac{1}{1 + T_{t\mathrm{Sensor}}s}
	\end{align}









